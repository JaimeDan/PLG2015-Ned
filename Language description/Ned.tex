\documentclass[10pt,a4paper]{article}
\usepackage[left=2cm,top=2.5cm,right=2.5cm,bottom=2.5cm]{geometry}
\usepackage[utf8]{inputenc}
\usepackage[spanish]{babel}
%\usepackage[pdftex]{graphicx}
\usepackage{amsmath}
\usepackage{amsfonts}
\usepackage{amssymb}
%\usepackage{enumerate}


%%%% g r a m m a r %%%%

%%%%%%%%%%%%%%

\newcommand{\cc}[1]{\texttt{#1}}
\newcommand{\ra}{\rightarrow}


\title{\huge Práctica PLG, el lenguaje Ned}
\author{Mayra Alexandra Castrosqui Florián\\
	Jaime Dan Porras Rhee
\date{Abril 2015}
}

%%%%%%%%%%%%%%%%%%%%%%%%%%%%%%%%%%%%%%%%%%%%%

\begin{document}


\maketitle
\abstract{En esta entrega se especificará el léxico y la sintaxis del lenguaje. Se expondrán ejemplos típicos de programas en dicho lenguaje.}



\subsection*{Descripción del lenguaje}
Ned es un lenguaje de programación imperativo. Un programa escrito en Ned se divide en cuatro bloques. Los tres primeros para declaraciones y el último que contiene el código a ejecutar. 
En el primero se declaran las variables globales. En el segundo bloque se declaran las clases con sus métodos y atributos. En el tercero se declaran funciones y procedimientos. Y en el último, rodeado por un \cc{Begin End} se encuentra el código principal.
Esquemáticamente sería de la forma 

\begin{verbatim}
Declare var
...
Declare class
...
Declare function
...
Begin
...
End
\end{verbatim}

\subsubsection*{Características del lenguaje}

Identificadores y ámbitos de definición \\
Clases, procedimientos y funciones. El lenguaje tendrá ámbitos de declaración dinámicos.\\

Tipos\\
Tipos con nombre y definición de tipos de usuario.\\

Instrucciones ejecutables\\
Instrucción case. Llamadas a procedimientos, funciones y métodos de clase.\\

Errores\\
Tratar de proseguir la compilación tras un error, a fin de detectar más errores.\\


\section{Análisis léxico}

Se tendrán las siguientes clases de unidades léxicas:
\begin{itemize}
\item Identificadores.
\item Palabras reservadas.
\item Constantes literales para los tipos int, float, char, string y bool.
\item Símbolos. 
\end{itemize}

Algunas palabras reservadas serán \cc{break}, \cc{continue}, \cc{const} entre otras que irán apareciendo en este documento.

\newpage

Los identificadores podrán admitir mayúsculas, minúsculas y ``\cc{\_}''. Los identificadores de variables y funciones empiezan por minúscula y los de clases por mayúscula.

\begin{equation*} %\label{eq1}
\begin{split}
id 	& \ra Min  (Min \mid May \mid Num\mid \cc{\_} )  ^{*} \\
Id	& \ra May  ( Min \mid May \mid Num \mid \cc{\_} )  ^*
\end{split}
\end{equation*}

Las constantes literales 

\begin{equation*}
\begin{split}
int 	& \ra Num^* \\
float 	& \ra Num^+.[Num^+] \\
char	& \ra \cc{`}c\cc{'} \\
string 	& \ra \cc{"{}} c^{*} \cc{"{}} \\
bool 	& \ra \cc{True}\mid \cc{False}
\end{split}
\end{equation*}


Auxiliares: 

\begin{equation*}
\begin{split}
Num	& \ra \cc{0}\mid ..\mid \cc{9} \\
Min	& \ra \cc{a}\mid ..\mid \cc{z}  \\
May	& \ra \cc{A}\mid ..\mid \cc{Z} \\
c 	& \ra Num\mid Min\mid May \mid Sym
\end{split}
\end{equation*}




\section{Análisis sintáctico}
El lenguaje $Ned$ se describe mediante una gramática incontextual $G$. $$G=(V_N, V_T, P, Code)$$
Las variables terminales son las unidades léxicas que obtenemos del análisis léxico. Las no terminales irán apareciendo en las reglas, además $Code \in V_N$. Veamos las reglas para la gramática de nuestro lenguaje.

\subsubsection*{Reglas}

Reglas para la estructura principal del código

\begin{equation*}
\begin{split}
Code 	& \ra Opc \ \cc{Begin}\ S \ \cc{End} \\
Opc 	& \ra [ \cc{Declare var} \ DV ] [ \cc{Declare class} \ DC ] [\cc{Declare function} \ DF ] \\
DV 	& \ra ( tipo (id\cc{;} \mid ASIG))^+ \\
DC 	& \ra ( \cc{Class} \ Id \ \cc{\{ public} \ (DF \mid DV)^* \  \cc{private} \ (DF \mid DV)^* \cc{ \}})^+ \\
DF 	& \ra (P \mid F)^+ \\
P 	& \ra id \ \cc{(}[tipo \ id \ (\cc{,} \ tipo \ id )^*]\cc{)}\cc{\{ }S\cc{ \}} \\
F 	& \ra tipo \  id \cc{(}[tipo \ id \ (\cc{,} \ tipo \ id )^*]\cc{)}\cc{\{ }S\cc{ \}} 
\end{split}
\end{equation*}

Reglas para las instrucciones

\begin{equation*}
\begin{split}
S 		& \ra  DV \mid ASIG \mid IF \mid WHILE \mid FOR \mid FOREACH \mid REPEAT \mid SWITCH \mid RET \\
ASIG 		& \ra id \cc{ = } VAL[ \cc{ [}Num\cc{] }] \cc{;} \\
IF 		& \ra \cc{if} \ B \ \cc{then} \ S \ \cc{else} \ S \ \cc{endif} \\
WHILE	& \ra \cc{while} \ B \ \cc{do} \ S \ \cc{endwhile} \\
FOR 		& \ra \cc{for ( }ASIG\cc{;}[ B(\cc{,}B)^* ]\cc{;} ASIG \cc{ )}\ S\ \cc{endfor} \\
FOREACH  	& \ra \cc{foreach}\ id\ \cc{in} \ id\ \cc{do} \ S \ \cc{endforeach} \\
REPEAT 	& \ra \cc{repeat} \ S \ \cc{until} \ B \ \cc{endrepeat} \\
SWITCH 	& \ra \cc{switch ( } id \cc{ ) } \ (\cc{case ( } VAL \cc{ ) } S)^+\ [\cc{Default} \ S] \ \cc{endswitch} \\
RET 		& \ra \cc{return}  \ VAL\cc{;}
\end{split}
\end{equation*}

Otras reglas

\begin{equation*}
\begin{split}
tipo 	& \ra \cc{int} \mid \cc{float} \mid \cc{char} \mid \cc{string} \mid \cc{bool} \\
VAL 	& \ra char \mid string \mid A \mid B \mid id
\end{split}
\end{equation*}

\newpage

Reglas para expresiones aritméticas y booleanas

\begin{equation*}
\begin{split}
A 	& \ra id \mid int \mid float \mid A\cc{+}A \mid A\cc{-}A \mid A\cc{*}A \mid A \cc{\^{}} A \mid A\cc{\%} A \mid A\cc{:}A \mid A\cc{/}A  \mid \cc{(}A\cc{)} \\
B 	& \ra bool \mid B \cc{\&\&} B \mid B\cc{||} B \mid  \cc{!}B \mid \cc{(}B\cc{)} \mid C \\
C 	& \ra A \ comp \ A \\
comp 	& \ra \cc{==}\mid \cc{!=} \mid \cc{<=} \mid \cc{>=} \mid \cc{<{}} \mid \cc{>{}}
\end{split}
\end{equation*}


Los símbolos de $comp$ son operadores infijos. Tienen todos la misma prioridad y no son asociativos. Con respecto a los operadores para la expresiones aritméticas y booleanas tenemos los siguientes cuadros. \\

%\begin{center}
\begin{tabular*}{0.80\textwidth}{ @{\extracolsep{\fill}}| c | c | c | c | }
  \hline
  \textsc{Operador} &  \textsc{Tipo} &  \textsc{Prio} & \textsc{Asociatividad} \\
  \hline 
   \cc{\^{}}  & binario infijo & 0 & asoc. a derechas    \\
  \hline
  \cc{*}  &binario infijo  & 1   & asoc. a izquierdas  \\
  \hline
  \cc{/}   & binario infijo  & 1   & asoc. a izquierdas   \\
  \hline
  \cc{:}  & binario infijo  & 2  & asoc. a izquierdas    \\
  \hline
  \cc{\%}  & binario infijo  & 3  & asoc. a izquierdas   \\
  \hline
  \cc{+}  & binario infijo & 4  & asoc. a izquierdas    \\
  \hline
  \cc{-}  & binario infijo & 4  & asoc. a izquierdas    \\
  \hline
\end{tabular*}
%\end{center}
\\

\begin{tabular*}{0.80\textwidth}{ @{\extracolsep{\fill}}| c | c | c | c | }
  \hline
  \textsc{Operador} &  \textsc{Tipo} &  \textsc{Prio} & \textsc{Asociatividad}\\
  \hline 
  \texttt{\&\&}  & binario infijo & 0 & asoc. a derechas    \\
  \hline
  \cc{||}  & binario infijo  & 1   & asoc. a izquierdas  \\
  \hline
  \cc{!}   & unario prefijo  &1   & asociativo   \\
  \hline
\end{tabular*}
\\


%\section{Ejemplos de código}
%\begin{lstlisting}
%Declare Var
%float r=3.14;
%
%Declare Class
%Class Racional{
%public 
%Rac
%
%Declare Function
%int mcd(int a, int b)
%int mcd = 1;
%for (i=1; i<min(a,b);i=i+1)
%if a\%i==0 && b\%i == 0 then
%if i>mcd then mcd = i;
%endif
%endif
%return mcd;
%
%int min(int a, int b)
%if a<b then return a;
%else return b;
%
%Begin
%Racional r1 = Racional(1,2);
%Racional r2 = Racional(1,3);
%r1=r1+r2;
%r1.diveq(mcd(r1.getnum, r1.getden));
%r=r1.tofloat()*r*2;
%end


%Declare funct 
%int factorial(int n)
%int fact = 1;
%for each i in {1..n}
%	fact = fact*i;
%end foreach
%return ract;
%
%Begin 
%int v=8;
%int resultado = factorial(v);
%End
%
%\end{lstlisting}



\end{document}